\documentclass[twoside,twocolumn]{article}

\usepackage{blindtext} % Package to generate dummy text throughout this template 
\usepackage{graphicx}
\usepackage[sc]{mathpazo} % Use the Palatino font
\usepackage[T1]{fontenc} % Use 8-bit encoding that has 256 glyphs
\linespread{1.05} % Line spacing - Palatino needs more space between lines
\usepackage{microtype} % Slightly tweak font spacing for aesthetics

\usepackage[english]{babel} % Language hyphenation and typographical rules

\usepackage[hmarginratio=1:1,top=32mm,columnsep=20pt]{geometry} % Document margins
\usepackage[hang, small,labelfont=bf,up,textfont=it,up]{caption} % Custom captions under/above floats in tables or figures
\usepackage{booktabs} % Horizontal rules in tables

\usepackage{lettrine} % The lettrine is the first enlarged letter at the beginning of the text

\usepackage{enumitem} % Customized lists
\setlist[itemize]{noitemsep} % Make itemize lists more compact

\usepackage{abstract} % Allows abstract customization
\renewcommand{\abstractnamefont}{\normalfont\bfseries} % Set the "Abstract" text to bold
\renewcommand{\abstracttextfont}{\normalfont\small\itshape} % Set the abstract itself to small italic text

\usepackage{titlesec} % Allows customization of titles
\renewcommand\thesection{\Roman{section}} % Roman numerals for the sections
\renewcommand\thesubsection{\roman{subsection}} % roman numerals for subsections
\titleformat{\section}[block]{\large\scshape\centering}{\thesection.}{1em}{} % Change the look of the section titles
\titleformat{\subsection}[block]{\large}{\thesubsection.}{1em}{} % Change the look of the section titles

\usepackage{fancyhdr} % Headers and footers
\pagestyle{fancy} % All pages have headers and footers
\fancyhead{} % Blank out the default header
\fancyfoot{} % Blank out the default footer
\fancyhead[C]{Virtualizacion y contenedores $\bullet$ Mayo 2019 $\bullet$ } % Custom header text
\fancyfoot[RO,LE]{\thepage} % Custom footer text

\usepackage{titling} % Customizing the title section

\usepackage{hyperref} % For hyperlinks in the PDF

%----------------------------------------------------------------------------------------
%	TITLE SECTION
%----------------------------------------------------------------------------------------

\setlength{\droptitle}{-4\baselineskip} % Move the title up

\pretitle{\begin{center}\Huge\bfseries} % Article title formatting
\posttitle{\end{center}} % Article title closing formatting
\title{Patrones de Diseño} % Article title
\author{Yofer Nain Catari Cabrera}
\date{\today} % Leave empty to omit a date
\renewcommand{\maketitlehookd}{%
\begin{abstract}
\noindent ...............................................
\end{abstract}
}

%----------------------------------------------------------------------------------------

\begin{document}

% Print the title
\maketitle

%----------------------------------------------------------------------------------------
%	ARTICLE CONTENTS
%----------------------------------------------------------------------------------------

\section{Introduccion}

\lettrine[nindent=0em,lines=3]{L}a ...............................................



%------------------------------------------------

\section{Objetivos}

\begin{itemize}
\item ...............................................
\item ...............................................
\item ...............................................
\item ...............................................
\item ...............................................


\end{itemize}




%------------------------------------------------

\section{Desarrollo}

\subsection{¿.................................?}

...................................................................................

\begin{center}
	\includegraphics[width=5cm]{./Imagenes/virtualizacion} 
	\end{center}




\subsection{¿........................?}

........................................................................\\
........................................................................\\
...........................................................................

\begin{itemize}
\item .....................................
\\ -------------------------------------------------------------
\\Usos de Docker
\\-------------------------------------------------------------------------------------------
\\ \textbf{-........................................................}


\begin{center}
	\includegraphics[width=5cm]{./Imagenes/docker} 
	\end{center}

Ventajas
\\ \textbf{- .......................................................}
 
\\ \textbf{- .......................................................}
Desventajas
\\ \textbf{- .......................................................}
\end{itemize} 

\subsection{----------------------------------------------------?}
------------------------------------------------------------------

\begin{itemize}
	\item .....................................................
	\\ \textbf{-App} .......................................
	\\ \textbf{-App} .....................................
	\begin{center}
	\includegraphics[width=5cm]{./Imagenes/jerarquia1} 
	\end{center}
\end{itemize} 

\begin{itemize}
	\item ..................................
.
	\\ \textbf{-App} .....................................
	\begin{center}
	\includegraphics[width=5cm]{./Imagenes/jerarquia2} 
	\end{center}
\end{itemize} 

\begin{itemize}
	\item -------------------------------------------------------------------
		\item -------------------------------------------------------------------
			\item -------------------------------------------------------------------
	\item -------------------------------------------------------------------



\end{itemize} 

\section{Conclusiones}

---------------------------------------------------------------------------------.
%----------------------------------------------------------------------------------------
%	REFERENCE LIST
%----------------------------------------------------------------------------------------

\begin{thebibliography}{99} % Bibliography - this is intentionally simple in this template

\bibitem[Martin, 2011]{Diego Martin:2011dg}
Martin, M.M,  y J.U (2011).
\newblock Virtualización, una solución para la eficiencia,
seguridad y administración de intranets
\newblock {\em El profesional de la informacion}, 350.
\newblock Contenedor de aplicaciones: Docker (2015)
 
 
\end{thebibliography}

%----------------------------------------------------------------------------------------

\end{document}
